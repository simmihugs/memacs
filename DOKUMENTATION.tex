% Created 2022-03-20 So 23:11
% Intended LaTeX compiler: pdflatex
\documentclass[a4paper]{article}
\usepackage[latin1]{inputenc}
\usepackage[T1]{fontenc}
\usepackage{graphicx}
\usepackage{longtable}
\usepackage{wrapfig}
\usepackage{rotating}
\usepackage[normalem]{ulem}
\usepackage{amsmath}
\usepackage{amssymb}
\usepackage{capt-of}
\usepackage{hyperref}
\usepackage{color}
\usepackage{listings}
\usepackage[english]{babel}
\usepackage[a4paper,left=1.5cm,right=1.5cm,top=2cm,bottom=2cm]{geometry}
\date{\today}
\title{DOKUMENTATION memacs}
\hypersetup{
 pdfauthor={},
 pdftitle={DOKUMENTATION memacs},
 pdfkeywords={},
 pdfsubject={},
 pdfcreator={Emacs 29.0.50 (Org mode 9.5.1)}, 
 pdflang={Germanb}}
\begin{document}

\maketitle
\tableofcontents
\section{Memacs - Gui Anwendungsprogrammierung}
\label{sec:org60d4a11}
Memacs ist eine monomer haskell app, f�r mehr infos �ber monomer
siehe die github seite von \href{https://github.com/fjvallarino/monomer}{monomer}, aber hier ist die grob
zusammenfassung.

\noindent Eine monomer app hat eine start funktion, die
\begin{itemize}
\item ein model (type),
\item events(type),
\item ui (function),
\item handleEvent (function) und
\item eine config erwartet.
\end{itemize}

\noindent Das model repr�sentiert die Daten, die Events k�nnen durch
user input bzw. Task oder Producer asynchron ausgel�st werden. Die
ui function definiert wie die GUI aussieht. Der Eventhandler,
spezifiziert was als folge auf welches Event passieren soll. Das
kann entweder sein das die Werte im Model angepasst werden, dann
w�rde die UI neu gerendert. Oder es werden asynchrone tempor�re oder
permanent laufende zusatz threads gestartet, die IO operationen
ausf�hren k�nnen. Die Config ist n�tig um dem Programm z.B. zugriff
auf Schriftarten zu geben - das merkt man ganz schnell wenn man
stack run in der src direcotry ausf�hrt und die monomer app leer
ist - weil keine Schriftarten existieren.

\subsection{Aufbau Memacs}
\label{sec:org539abce}
Bei Memacs habe ich den Ansatz verfolgt, den code in mehrere
logische einheiten zu unterteilen.

\noindent Es gibt die UI.* module, das sind alle ui functions die,
geb�ndelt in MainUI.hs zusammengefasst werden. Analog gilt das
gleiche f�r Events und MainEvents.hs. Zus�tzlich gibt es Types.*
darin sind allgemein genutzte typen, Calendar und Task
definiert. Und dann gibt es noch Utils f�r alle m�glichen helper
functions. Finale haben wir noch das submodule Parser - darin ist
der ParserCombinator Parser definiert und meine QuasiQuotes und das
submodule Widget - worin sich das tetrismodule befindet - meine
eigene implementation eines Monomer widgets.

\noindent Memacs hat eine Sidebar, die einen von 4 Bereichen per
Klick auf den Icon darstellen kann. File/Write, Calendar/Tasks,
Tetris und Settings. F�r die Bereiche sind verschiedene T�tigkeiten
in das Projekt eingeflossen.
\begin{itemize}
\item Settingsbereich
Parsen von Json
\item Calendar
Definieren von Calendar datatype, und Task datatyp, einlesen von
xml und schreiben von xml
\item Tetris spiel
\item commandoZeile (Alt-x, Esc, Enter)
\item Filebrowser und Texteditor
Es gibt einen Filetree aus dem heraus dateien geoeffnet werden
koennen, direcotries gewechselt werden koennen
\end{itemize}
\subsection{Prozess und Probleml�sung}
\label{sec:orge9f1ff2}
Basierend auf der hackage dokumentation von Monomer bin ich schnell
voran gekommen viel code zu schreiben, der innerhalb von k�rzester
Zeit kaum noch zu maintainen war - und refactoring hat das ganz
unkompilierbar gemacht - also war ich gezwungen das projekt nochmal
anzufangen. Die Erfahrung hat mich leider nicht gelehrt - und das
gleich ist noch mal passiert. Der Hauptgrund wieso in der aktuellen
form so viele so kleine Module existieren, die �bersichtlichkeit
gegen�ber weniger sehr gro�en. Zu mindest f�r mich.

\noindent Neben dem blo�en bauen von UI elementen habe ich
basierend auf den tutorial in der Monomer mein eigenes widget -
TetrisWidget angelegt

\noindent Urspr�nglich war ich dabei das TextArea widget
anzupassen, damit man syntax highlighting und line numbers hat
usw. aber zeitlich und vom Verst�ndnis her bin ich da an meine
grenzen gesto�en. Ich habe mich darauf beschr�nkt,
syntaxhighlighting f�r Haskell files hinzuzuf�gen. Meine �nderungen
sind in Wiget/My* enthalten.

\noindent Eine Funktion die ich in Memacs eingebaut habe, und die
wohl etwas zu wenig effektiv genutzt wird, ist ein minimales log
system, das erlaubt bei Events eine Nachricht zu loggen. Die Idee
war grob die Ursache von Fehlern zu finden.

\noindent Am Effektivsten um Problemen vorzubeugen und diese zu
l�sen, ist f�r mich nun die positive erfahrung mit der Aufteilung
von meinem Code in kleinere Module, die wesentlich �bersichtlicher
sind und weniger unklarheit zulassen welche �nderung und wo den
fehlern introducted hat.
\section{Funktionale Referenzen (Linsen)}
\label{sec:org72afcfd}
\subsection{Prozess und Probleml�sung}
\label{sec:org1ef0e4d}
Begr�ndet durch die Gui library monomer werden linsen praktisch
�berall verwendet. Zum einen weil permanent Werte ben�tigt werden
bzw. ver�ndert werden als reaktion auf ein Event in der Gui. Und
zum anderen weil die Widgets in Monomer i.d.R. eine Linse als input
annehmen. Ein Textfield bekommt eine Linse zu Text im Model und
wenn der Text im TextField angepasst wird, wird der Wert im Model
durch die Linse angepasst.
\subsection{Relevanten Bereiche im src code}
\label{sec:org5a0d3eb}
\begin{itemize}
\item Erzeugung von Linsen ist vorallem in Model.hs und die submodules
in Model/*
\item Die Verwendung von Linsen findet statt in
\begin{itemize}
\item UI/* und Widget/* um Daten aus dem Model zu beziehen
\item Events/*
\end{itemize}
\end{itemize}
\section{Template Haskell}
\label{sec:org205b0bc}
\subsection{Prozess und Probleml�sung}
\label{sec:org60480c4}
Allgemein k�nnte der Prozess mit sehr viel Frustation und die
L�sung mit Dankbarkeit umrissen werden. Nach dem die Vorlesung
schon ein bisschen her war, habe ich erstmal das skript nochmal
gelesen. Das hat mir leider nicht gereicht um wirklich weiter zu
kommen. Dann habe ich mich ins Internet gest�rzt und ein tutorial,
blockpost youtube video nach dem n�chsten gesucht um zu raffen wie
das mit quasiquotern funktionieren kann. Ich denke ich hab bestimmt
2-3 Tage gebraucht um irgendwas sinnvolles im zusammenhang von
quasiquoter hinzubekommen.

\noindent Ich wollte dann quasiquoter benutzen um Commandos, die
ein user in memacs in die eingabezeile eintippt per quasiquoter,
auszuf�hren, �ber mehrere videos bin ich dann auf das Thema Parser
combinators gekommen und dann gabe es auf einmal auch eine
Verbindung zu quasiquotern. Und basierend von einem Blockpost (Im
source code ist die quelle zitiert) habe ich dann einen Parser und
darauf basierend meinen Typ Open implementiert. Und zu Open habe
ich dann den quasiquoter openQQ geschrieben. Damit kann man im
source code einen Quasiquoter benutzen um einen string zum Open typ
parsen \ldots{} der Variablen hat, die im quasiquoter von der umgebung
eingesammelt werden \ldots{}

\noindent Dann ist mir aber aufgefallen, dass man einen quasiquoter
schreiben kann, der direkt einen eingabe string parsed - und daraus
ist dann der banale quasiquoter settingsQQ geworden, der es
moeglich macht basierend auf dem Parser json daten direkt in einen
JValuewert zu parsen. Und dass ist in memacs jetzt die
Implementation der defaultsettings.

\subsection{Relevanten Bereiche im src code}
\label{sec:org8823841}
\begin{itemize}
\item Utils/Style.hs
\item Parser/Parser.hs
\item Parser/Json.hs
\item Parser/QQ.hs
\item Parser/Open.hs
\end{itemize}
\section{Build instructions}
\label{sec:orgc801301}
Informationen zu den allgemeinen dependencies von monomer kann die
github seite 
\url{https://github.com/fjvallarino/monomer/blob/main/docs/tutorials/00-setup.md}
gelesen werden.

\noindent Memacs benutzt
\url{https://github.com/fjvallarino/monomer-starter} als Basis (benutzt
aber keinen Haskell code aus der Basis)
\end{document}